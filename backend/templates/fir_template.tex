% fir_template.tex -- Compile with XeLaTeX
\documentclass[11pt,a4paper]{article}
\usepackage{geometry}
\geometry{margin=0.9in}
\usepackage{fontspec}
\usepackage{polyglossia}
\setmainlanguage{english}
\setotherlanguage{marathi}
\newfontfamily\devanagarifont[Script=Devanagari]{FreeSerif}
\newfontfamily\englishfont[Scale=1.0]{TeX Gyre Termes}
\usepackage{array}
\usepackage{tabularx}
\usepackage{multirow}
\usepackage{booktabs}
\usepackage{setspace}
\usepackage{titlesec}
\usepackage{enumitem}
\setlist{nosep}
\parindent0pt
\renewcommand{\arraystretch}{1.25}

\begin{document}

% HEADER
{\small\hfill\textbf{N.C.R.B. (एन.सी.आर.बी.)}}\\[-2pt]
{\centering\textbf{I.I.F.-I (पहिली माहिती अहवाल - १)}\par}
{\centering\textbf{FIRST INFORMATION REPORT}\par}
{\centering\textit{(Under Section 173 B.N.S.S.)}\par}
{\centering\devanagarifont\textbf{थम खबर अहवाल}\par}
{\centering\devanagarifont(बी एन एस एस संदर्भात कलम 173 अंतर्गत)\par}

% TABLE 1
\noindent
\begin{tabularx}{\textwidth}{@{}p{0.45\textwidth}X@{}}
\textbf{1. District (जिल्हा):} & \devanagarifont {{district}} \\
\textbf{P.S. (पोलीस ठाणे):} & \devanagarifont {{policeStation}} \\
\textbf{Year (वर्ष):} & {{year}} \\
\textbf{FIR No. (थम खबर क्र.):} & {{firNo}} \\
\textbf{Date and Time of FIR (दि. व वेळ):} & {{firDate}} \quad {{firTime}}
\end{tabularx}

% TABLE 2 (Acts / Sections)
\noindent
\textbf{2.} \quad \textbf{S.No. (अ.क्र.)} \hspace{2cm}
\textbf{Acts (अधिनियम)} \hspace{2cm}
\textbf{Sections (कलम)} \\
\rule{\textwidth}{0.4pt}\\[-4pt]
{{sections}}

% OCCURRENCE
\noindent
\textbf{3. (a) Occurrence of offence (गुन्ह्याची घटना):}
\begin{tabularx}{\textwidth}{@{}p{0.45\textwidth}X@{}}
\textbf{Day (दिवस):} & \devanagarifont {{occurrenceDay}} \\
\textbf{Date from (दिनांक पासून):} & {{occurrenceDateFrom}} \\
\textbf{Date to (दिनांक पर्यंत):} & {{occurrenceDateTo}} \\
\textbf{Time Period (कालावधी):} & \devanagarifont {{occurrenceTimePeriod}} \\
\textbf{Time From (वेळेपासून):} & {{occurrenceTimeFrom}} \\
\textbf{Time To (वेळेपर्यंत):} & {{occurrenceTimeTo}}
\end{tabularx}

% Info received at PS
\noindent
\textbf{(b) Information received at P.S. (पोलीस ठाण्यावर माहिती मिळाल्याचा):}
\begin{tabularx}{\textwidth}{@{}p{0.45\textwidth}X@{}}
\textbf{Date (दिनांक):} & {{infoReceivedDate}} \\
\textbf{Time (वेळ):} & {{infoReceivedTime}}
\end{tabularx}

% General Diary Reference
\noindent
\textbf{(c) General Diary Reference (ठाणे दैनिक संदर्भ):}
\begin{tabularx}{\textwidth}{@{}p{0.45\textwidth}X@{}}
\textbf{Entry No. (नोंद क्र.):} & {{gdEntryNo}} \\
\textbf{Date and Time (दिनांक व वेळ):} & {{gdDateTime}}
\end{tabularx}

% TYPE OF INFORMATION
\noindent
\textbf{4. Type of Information (माहितीचा प्रकार):} \devanagarifont {{typeOfInfo}}

% PLACE OF OCCURRENCE
\noindent
\textbf{5. Place of Occurrence (घटना ठिकाण):}
\begin{tabularx}{\textwidth}{@{}p{0.45\textwidth}X@{}}
(a) Direction and distance from P.S. (पोलीस ठाण्यापासून दिशा व अंतर): & \devanagarifont {{directionDistanceFromPS}} \\
(b) Address (पत्ता): & \devanagarifont {{address}} \\
(c) District (State) (जिल्हा (राज्य)): & \devanagarifont {{districtState}}
\end{tabularx}

% COMPLAINANT
\noindent
\textbf{6. Complainant (तक्रारदार):}
\begin{tabularx}{\textwidth}{@{}p{0.45\textwidth}X@{}}
Name: & {{complainantName}} \\
Father/Husband Name: & {{complainantFatherOrHusbandName}} \\
DOB: & {{complainantDOB}} \\
Nationality: & {{complainantNationality}} \\
UID No.: & {{complainantUIDNo}} \\
Passport No.: & {{complainantPassportNo}} \\
Occupation: & {{complainantOccupation}} \\
Current Address: & {{complainantCurrentAddress}} \\
Permanent Address: & {{complainantPermanentAddress}} \\
Phone: & {{complainantPhone}} \\
Mobile: & {{complainantMobile}}
\end{tabularx}

% ACCUSED
\noindent
\textbf{7. Accused (आरोपी):}
{{accusedList}}

% PROPERTIES OF INTEREST
\noindent
\textbf{8. Properties of Interest (मालमत्ता):}
{{propertiesList}}

% TOTAL VALUE
\noindent
\textbf{9. Total Value of Property:} {{totalValueOfProperty}}

% FIRST INFORMATION CONTENTS
\noindent
\textbf{10. First Information Contents:}
{{firstInformationContents}}

\end{document}
